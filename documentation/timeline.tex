\documentclass{article}
\usepackage[utf8]{inputenc}
\usepackage{hyperref}

\title{Share-IITK(Backend)}
\author{\href{mailto:abhays@iitk.ac.in}{\textit{Abhay Pratap Singh}}
		\\
        \href{mailto:parv@iitk.ac.in}{\textit{Parv Mor}}}
\date{June 2017}

\begin{document}

\maketitle
\section*{Team Members:}
\begin{itemize}
	\item Abhay Pratap Singh
    \item Parv Mor
\end{itemize}
\section*{Important Links:}
\begin{itemize}
	\item \href{https://github.com/abhayptp/share-iitk}{Share-IITK repo}
\end{itemize}
\section*{\textbf{Timeline}}
\subsection*{1\textsuperscript{st} week:}
\begin{itemize}
	 \item Started learning scala from this book \href{https://www.manning.com/books/functional-programming-in-scala}{Functional Programming in Scala}
    \item Jumped to this link \href{https://twitter.github.io/scala_school/}{Twitter Scala School}, as I found it more useful.
    \item Came across \href{http://danielwestheide.com/scala/neophytes.html}{Neophytes Guide to Scala}, studied part - 8 from there, which was about concurrent side of scala.
\end{itemize}

\subsection*{2\textsuperscript{nd} week:}
\begin{itemize}
	\item Learned about sbt build tool.
    \item Read about the features of REST API.
    \item Decided the endpoints for the API. There will be basically four endpoints:
    	\begin{itemize}
    		\item /api/resources/search(GET): It will return the whole table(in JSON) to be searched in the frontend.
            \item /api/resources/upload(POST): For uploading the respective file.
            \item /api/resources/(GET): For downloading the file, distinguished by parameter MD5. Also, it will increment the score of that file by 1.
            \item /api/courses/(GET): Return the list of courses.
     	\end{itemize}
    \item Learned about Actors in Akka and how they are used for asynchronous calls.
	\end{itemize}
\subsection*{3\textsuperscript{rd} week:}
\begin{itemize}
	\item Followed some links to get an idea of akka http REST API, for example:
    	\begin{itemize}
        	\item \href{https://spindance.com/reactive-rest-services-akka-http/}{Reactive REST Services using Akka HTTP}
            \item \href{https://dzone.com/articles/building-rest-service-scala}{Building a REST Service in Scala with Akka HTTP, Akka Streams and Reactive Mongo - DZone Java}
            \item \href{https://danielasfregola.com/2016/02/07/how-to-build-a-rest-api-with-akka-http/}{How to build a REST API with Akka Http}
	\end{itemize}
	\item Wrote the code for handling HttpResponses using akka for \href{https://github.com/abhayptp/share-iitk}{Share IITK} repo.
    
\end{itemize}
\end{document}
